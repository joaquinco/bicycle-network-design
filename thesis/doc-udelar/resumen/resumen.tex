\begin{abstract}
    La modalidad de transporte urbano utilizando bicicleta puede traer beneficios tanto en la salud y bienestar de las personas como para la ciudades disminuyendo la congestión del tráfico, mejorando la calidad del aire y reduciendo la contaminación sonora. Quienes están a cargo de las decisiones sobre la gestión del espacio público en ciudades (municipalidades, intendencias) pueden realiza acciones para que más personas realicen sus viajes habituales en bicicleta. Un tipo de acción es la construcción de infraestructura para circular en bicicletas, también conocidas como ciclovías. Las infraestructuras se pueden construir sobre la red de calles de una ciudad formando una red de ciclovías, estas pueden ser de varios tipos o tecnologías y afectan la experiencia de usuario de manera que a mejor experiencia mayor costo de construcción tienen. Asumiendo que se conoce la demanda origen-destino potencial de viajes en bicicleta que actualmente utiliza otro medio de transporte, que se cuenta con un presupuesto acotado para construir ciclovías y que más personas van a utilizar bicicleta si cuentan con más y mejor infraestructura específica de ciclovías, en esta tesis estudiamos el problema de maximizar la atracción de demanda hacia la bicicleta desde otros medios de transporte mediante la decisión de dónde y qué tipo de ciclovías construir. Lo modelamos como un problema de optimización en redes, mediante una formulación de programación matemática no lineal que luego reformulamos para obtener una formulación lineal entera mixta (MILP). El problema resultante se resuelve con técnicas de resolución MILP del estado del arte que probamos numéricamente con una instancia tomada de la literatura y una construida en base a datos de la ciudad de Montevideo.
\end{abstract}
