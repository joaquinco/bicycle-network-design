\chapter{Detalles de implementación}

Durante el transcurso del proyecto desarrollamos una biblioteca escrita en Python \footnote{\url{https://gitlab.fing.edu.uy/joaquin.correa/bicycle-network-design}} que facilita la manipulación y especificación de datos y la ejecución de las diferentes formulaciones sobre diferentes solvers. Esto permitió simplificar la tarea de ejecución y manipulación de parámetros significativamente, por ejemplo alterar los parámetros de una instancia como la red subyacente o la demanda, generar instancias aleatorias sobre una red o aplicar el problema sobre grafos obtenidos de OpenStreeMaps \footnote{\url{https://www.openstreetmap.org/}}. Especificamos un formato de salida unificado que simplificó el análisis de las soluciones por medio de la interpretación de los archivos de solución de los diferentes solvers. Esto nos permitió comparar rápidamente un gran volumen de soluciones así como la representación gráfica de los resultados.

La formulación del problema la escribimos en el lenguaje GNU MathProg \footnote{\url{https://lpsolve.sourceforge.net/5.5/MathProg.htm}} debido a su simpleza y expresividad. Utilizamos tres solvers durante el proyecto: GLPK \footnote{GNU Linear Programming Kit}, CBC \footnote{\ Coin-or branch and cut MIP solver} y AMPL/CPLEX \footnote{IBM CPLEX Optimizer}. Los primeros dos son libres y de código abierto y fueron utilizados en las primeras etapas del proyecto. GLPK fue tomado como punto de partida y referencia debido a su estabilidad y utilización en el área. CBC demostró ser más rápido que GLPK gracias a que saca provecho de procesamiento paralelo, pero su utilización fue ligeramente más complicada debido a que requiere compilación manual con la extensión de GNU MathProg y en su salida no notifica cuando una instancia del problema es no factible. Luego, para bajar los tiempos de ejecución sobre instancias más complejas utilizamos el sofware comercial AMPL/CPLEX para ejecutar las pruebas del Capítulo \ref{sect:problemresults} Experimentos Computacionales. El solver CPLEX es el estado del arte en resolución de problemas MILP y dispusimos de una licencia académica.

\section*{Datos de la red}
\label{sect:costcalculation}

Las redes deben tener una serie de datos asociados de manera que sea posible solucionarlas con la biblioteca desarrollada. Para los nodos es necesario (no estrictamente) tener su par de coordenadas. Para los arcos, se necesitan los siguientes atributos:

\begin{description}
\item[user\_cost (abreviado CU)]: Costo de usuario de atravesar el arco sobre el grafo base (sin infraestructura o con la tecnología base $i_0$).
\item[construction\_cost (abreviado CC)]: Costo de construcción de la tecnología 1.
\end{description}

Luego, si se tienen los tipos de tecnologías $I = \{1, 2, 3, ... \}$, para cada arco, se calcula el costo de usuario de atravesarlo utilizando la tecnología $i$ como $CU_i = {28 - 3 (i + 1) \over 25}$. Además, el valor del costo de construcción será $CC_i = 2i CC$. De ser necesario, pueden especificarse valores particulares de costos de usuario y construcción para cada arco y tecnología.
