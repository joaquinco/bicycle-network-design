\chapter{Validación de la formulación binivel}
\label{sect:apendixbilevelvalidation}

Para discutir la validez matemática del modelado binivel (\ref{eq:objective1lvl})-(\ref{eq:flowbalance}), definida en el Capítulo \ref{sect:problemdefinition}, partimos de una formulación estándar de BLPP lineal, como se puede encontrar en Bard \cite{bardbook} y demostramos la validez en base a las definiciones ahi mencionadas justificando según nuestra formulación. Sea la siguiente formulación simplificada del problema binivel:

\begin{align}
\text{max}_{y \in Y}    & \; F(x, y) \label{bilevelgeneric1} \\
\text{s.t.} \modelspace & A_1 x + B_1 y \leq b_1 \\
                        & y \geq 0 \\
                        & \text{min}_{x \in X} f(x, y) \\
                        & \modelspace A_2 x + B_2 y \leq b_2 \label{bilevelgeneric5} \\
                        & \modelspace x \geq 0 \label{bilevelgeneric6}
\end{align}

Y las siguientes definiciones:

\begin{definition}
Conjunto factible
\begin{align}
  S = \{(x, y) \setminus x \in X, y \in Y, A_1 x + B_1 y \leq b_1, A_2 x + B_2 y \leq b_2 \}
\end{align}
\end{definition}

\begin{definition}
Conjunto de reacción del segundo nivel:
\begin{align}
  P(y) = \{ x \in \text{argmin}_{\hat{x} \in X} f(\hat{x}, y) : A_2 \hat{x} + B_2 y \leq b_2 \}
\end{align}
\end{definition}

Diremos que el problema está bien formulado si el conjunto $S$ es no vacío, es decir, si existen soluciones factibles y se cumple que para toda $y$ el conjunto $P(y)$ es no vacío, en otras palabras, si para todo movimiento del problema de primer nivel, hay margen de decisión en el segundo nivel.

\begin{lemma}$S$ es no vacío
\end{lemma}

\begin{proof}
$S \neq \emptyset$ ya que $\exists (x_0, y_0) \in X \times Y$ donde $y_0$ es el vector $y_{ai_0} = 1 \forall a \in A$, $i_0$ es la tecnología cuyo costo $H_{ai_0} = 0$, lo que deja al resto de las entradas del vector $y_0$ en $0$. Por lo tanto se cumple la restricción (\ref{eq:alwaysoney}) dado que todos los arcos tienen infraestructura activa, y la restricción (\ref{eq:respectbudget}) dado que el costo total de construcción de $y_0$ es $0$.

Luego, dado que la infraestructura de ciclovía activa logra la conectividad de los pares origen-destino y el hecho de que el costo de los arcos $C_ai$ sea no negativo, posibilita asegurar que el problema de segundo nivel tiene al menos una solución factible $x_0$.
\end{proof}

\begin{lemma}$\forall y \in Y,\; P(y) \neq \emptyset$
\end{lemma}

\begin{proof}
Para cualquier asignación $y = \hat{y} \in Y$, se cumple que $P(\hat{y})$ es no vacío ya que todos los arcos tienen infraestructura activa, por lo tanto el grafo donde los flujos del problema de segundo nivel pueden pasar es conexo y llega necesariamente a todos los nodos, incluyendo los pares origen-destino. Por lo tanto el espacio de soluciones factibles del subproblema es no vacío. 
\end{proof}

Para nuestro problema en particular se cumple que $P(y)$ puede no ser univaluado en términos de flujos ya que entre cada par origen destinos pueden haber varios caminos más cortos. Sin embargo como el problema de primer nivel solo considera el costo de los caminos entonces esta consideración pierde relevancia. Sin embargo debe ser tenido en cuenta porque en alguna aplicación puede ser necesario considerar otros aspectos de los flujos ademas de su costo.
