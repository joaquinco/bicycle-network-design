\chapter{Conclusiones y trabajo a futuro}

Proponemos una nueva variante del problema de diseño de redes de ciclovías considerando varios tipos de tecnologías de ciclovías, que implican diferentes costos de construcción y costos de usuario. El modelo tiene la particularidad que la no construcción es ciclovías es una solución factible que consideramos razonable y realista. Permitimos modelar el comportamiento de la demanda de manera que pueda adaptarse a diferentes realidades mediante la especificación de funciones de transferencia de demanda particulares. La demanda considerada puede transferirse a la bicicleta desde otros modos y esta transferencia ocurre si la ciclovía construida permite a los usuarios circular por caminos de menor costo en relación con la situación inicial. Lo abordamos como un problema de optimización utilizando como marco conceptos consolidados en área de Investigación Operativa como el diseño de redes {\it Multi-Facility, Multi-Commodity} y {\it piecewise linear costs}. Finalmente, lo resolvimos con el solver CPLEX que corresponde al estado del arte en tecnología de resolución de problemas MILP. Todas estas características en conjunto lo diferencian de otros trabajos en los que ya se han abordado algunos de estos puntos.

Durante el desarrollo del trabajo consideramos varias alternativas para su formulación: binivel, MILP de un único objetivo y MILP multiobjetivo. Nos decidimos por el último, ya que buscábamos un método de resolución exacto que sea aplicable a instancias de características realistas. Pese a la complejidad de nuestro modelo, este logra resolver correctamente instancias de tamaño medio en tiempo razonable aunque la instancia de la red de Montevideo junto a los parámetros utilizados estaba en el límite del tamaño manejable por el solver utilizado, dado que cualquier incremento en cantidad de tipos de tecnologías o cantidad de pares origen-destino disparaban los tiempos de ejecución o utilización de recursos más allá de lo disponible para este trabajo. Sin embargo, dado que el problema es de planificación estratégica a largo plazo, en una aplicación real el tiempo de ejecución no es tan determinante y se podrían asignar recursos excepcionales llegada la necesidad, sin olvidar que estamos trabajando sobre un problema de complejidad NP.

Si bien el modelo fue pensado para el diseño de ciclovías desde cero, puede contemplar el problema de mejora de una red de ciclovías asignando un costo de construcción cero a la tecnología que se encuentre construida. Esto es posible dado que modelamos los costos por arco y tecnología pese a que luego especificamos los parámetros de manera general solamente en función del tipo de tecnología y el largo del arco.

Mediante la realización de experimentos numéricos en una instancia chica y una instancia de la ciudad de Montevideo, basados en datos fundados, mostramos que es posible construir una red de ciclovías que induce una buena medida de demanda transferida a la bicicleta incluso con lo que se considera bajos factores de presupuesto. De nuestras pruebas concluimos que utilizar mayor cantidad de puntos de quiebre favorece tanto la demanda transferida total como decisiones más inteligentes en términos de infraestructuras construidas a razón de un mayor costo computacional. También concluimos que la función de transferencia de demanda que se utilice es relevante en los resultados en varios aspectos además de la demanda transferida, por ejemplo cantidad de pares origen destino afectados y factor de desvío de los caminos de menor costo de usuario sobre los caminos de menor distancia.

Como variantes posibles, pensamos que si bien permitimos modelar funciones de transferencia de demanda arbitrarias, vale la pena mencionar que se puede desarrollar una versión más eficiente únicamente para la función de transferencia lineal, mediante la representación explícita de la función de transferencia en la función objetivo sin utilizar el mecanismo de los puntos de quiebre. Además, esta versión no necesitaría ser multiobjetivo dado que la representación de la función de transferencia sería estrictamente decreciente. Otro enfoque para modelar la demanda que se transfiere a la bicicleta es mediante los modelos de elección discreta siguiendo el trabajo \parencite{Pacheco2021}, cambiando la función objetivo para que maximice la cantidad esperada de utilización de la bicicleta. Además, se pueden contemplar otros aspectos en los costos o utilidad percibida por los usuarios respecto a los trayectos en bicicleta, por ejemplo la distancia total y distancia fuera de la red de ciclovías. Otros conceptos con los que podemos extender nuestro modelo son: contemplar las discontinuidades en los tipos de tecnologías \parencite{baya2021} durante un trayecto en bicicleta, integrar el concepto de camino seguro de manera que la demanda entre un origen y un destino pueda potencialmente ser satisfecha solo si existe una ciclovía que los une \parencite{Duthie2014}, y permitir distinguir qué tipos de tecnologías se puede construir en cada arco de la red, aspecto que puede ser necesario para contemplar las diferentes realidades dentro de una ciudad. También se podría adaptar el modelo a un enfoque de bicicletas compartidas, realidad que es común en otras partes del mundo y requiere tener en cuenta los centros de estacionamientos requeridos para este sistema \parencite{vogel2016}.

Entendemos que seguir agregando complejidad al modelo puede resultar en que sea inaplicable debido a los altos tiempos de ejecución, dado que en la situación actual ya nos encontramos con un límite en el generoso hardware con el que dispusimos. Identificamos dos alternativas para circunvalar esta situación: solucionar el problema de manera aproximada con una metaheurística o reestructurar la formulación de manera de aplicar el algoritmo de descomposición de Benders \parencite{bucarey2022, crainic2021} lo que puede mejorar los tiempos de ejecución de la resolución exacta.

Respecto a los datos, en este trabajo estimamos y aproximamos muchos de los parámetros. Puede ser interesante utilizar datos más precisos para este problema, por ejemplo, una matriz de demanda que considere varios modos de transporte. Por otro lado, podemos comparar las soluciones del modelo en términos de demanda transferida contra un escenario real de dos formas: la más costosa es mediante encuestas, suponiendo que cierta infraestructura es construida, enfoque similar a \parencite{shwe2014}; la otra es contrastar los resultados del modelo contra datos de ciudades que mantengan un histórico sobre la utilización de la bicicleta y cobertura de ciclovías.
