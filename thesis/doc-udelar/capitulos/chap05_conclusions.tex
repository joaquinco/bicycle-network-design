\chapter{Conclusiones y trabajo a futuro}

En este trabajo proponemos una nueva variante del problema de diseño de ciclovías. Consideramos varios tipos de tecnologías y permitimos modelar el comportamiento de la demanda de manera que pueda adaptarse a diferentes realidades mediante la especificación de funciones de transferencia de demanda particulares. Esta última característica es la principal diferencia con trabajos anteriores y puede ser considerado el principal aporte a la comunidad académica.

Durante el desarrollo del trabajo consideramos varias alternativas para su formulación: binivel, MILP de un único objetivo y MILP multiobjetivo. Nos decidimos por un el último que pese a su complejidad logra resolver correctamente instancias de tamaño medio en tiempo razonable aunque la instancia de Montevideo y parámetros utilizados ya estaban en el límite del tamaño manejable dado que cualquier incremento en cantidad de tipos de tecnologías o cantidad de pares origen-destino disparaban los tiempos de ejecución más alla de lo permitible para este trabajo. Sin embargo, dado que el problema es de planificación estratégica a largo plazo, en una aplicación real el tiempo de ejecución no es tan determinante.

Si bien el modelo fue pensado para el diseño de ciclovías desde cero, puede contemplar el problema de mejora de una red de ciclovías asignando un costo de construcción cero a la tecnología que se encuentre construida. Esto es posible dado que modelamos los costos por arco y tecnología pese a que luego los especificamos de manera general solo en función del tipo de tecnología y el largo del arco.

Nuestro problema puede ser considerado de alta complejidad al ser un diseño de redes {\it Multi-Facility, Multi-Commodity} con atracción de demanda gradual. Tiene la particularidad que la no construcción es ciclovías es una solución factible que consideramos razonable y realista. Todas estas características en conjunto lo diferencian de otros trabajos en los que ya se han tocado algunos de estos puntos.

Mediante la realización de experimentos numéricos en una instancia chica y una instancia de la ciudad de Montevideo, basados en datos fundados, mostramos que es posible construir una red de ciclovías que induce una buena medida de demanda transferida a la bicicleta incluso con lo que se considera bajos factores de presupuesto.

A futuro, identificamos algunos caminos de trabajo. En primer lugar, extender la formulación propuesta en este trabajo para que considere otros aspectos manejados en otros trabajos como discontinuidades en los tipos de tecnologías \cite{baya2021}, integrar el concepto de camino seguro de manera que la demanda entre un origen y un destino pueda potencialmente ser satisfecha si existe una ciclovía que los une \cite{Duthie2014}, y permitir distinguir qué tipos de tecnologías se puede construir en cada arco de la red que puede ser necesario para contemplar las diferentes realidades dentro de una ciudad.

También se podría adaptar el modelo a un enfoque de bicicletas compartidas, realidad que es común en otras partes del mundo y requiere tener en cuenta los centros de estacionamientos requeridos para este sistema \cite{vogel2016}.

Entendemos que seguir agregando complejidad al modelo puede resultar en que sea inaplicable debido a los altos tiempos de ejecución dado que en la situación actual ya nos encontramos con un límite en el generoso hardware con el que dispusimos. Identificamos dos alternativas para circunvalar esta situación: solucionar el problema de manera aproximada con una metaheurística ó reestructurar el problema y utilizar descomposición de Benders \cite{bucarey2022}, \cite{Crainic2021}.

Respecto a los datos, en este trabajo estimamos y aproximamos muchos de los parámetros. Pensamos que puede ser interesante aplicar este problema con datos específicos para este problema. Luego podemos comparar las soluciones del modelo en términos de demanda transferida contra un escenario real de dos formas: la más costosa sería mediante encuestas suponiendo que cierta infraestructura es construida, similar a \cite{shwe2014}; la otra sería contrastar los resultados del modelo contra ciudades que mantengan algún histórico de datos de utilización de la bicicleta y cobertura de ciclovías.
