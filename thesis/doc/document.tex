\documentclass{article}
  \parindent = 0mm % Sin sangría
  \usepackage[utf8]{inputenc}
  \usepackage[T1]{fontenc}
  \usepackage[spanish]{babel}
  \usepackage{graphicx}
  \usepackage{amstext}
  \usepackage{amsmath}
  \usepackage{amsthm}
  \usepackage{booktabs}
  \usepackage{subfigure}
  \usepackage{footnote}
  \usepackage{hyperref}
  \usepackage{algpseudocode,algorithm,algorithmicx}
  \usepackage[font=small,labelfont=bf]{caption}
  \usepackage{esvect}

  \newtheorem{definition}{Definición}
  \newtheorem{lemma}{Lema}

  \newcommand{\modelspace}{\hspace{1.5em}}

\begin{document}
  \begin{center}
    {\sc \large Maestría en Investigación de Operaciones}
    
    {\sc \large Tesis - Borrador}
    \linebreak

    {\rm Joaquín Correa - \today}
  \end{center}

  \section*{Problema}

  \subsection*{Descripción}

  El problema se ubica dentro del contexto de diseño y planificación de ciclovías en una ciudad.

  Dada una red que modela una ciudad y una matriz origen-destino con demanda asociada, el objetivo es maximizar la cantidad de demanda que utiliza la bicicleta como modo de transporte. Para poder lograr esto, se cuenta con infraestructuras y presupuesto, que de ser utilizados, permiten disminuir el costo percibido por el usuario al utilizar la bicicleta, asumiendo que utiliza el camino más corto, y por lo tanto potencialmente influir la decisión de utilizarla como modo de transporte.

  \subsection*{Formulación}

  En este trabajo se parte de una formulación binivel del problema dado que es una representación directa de lo que se quiere resolver. El primer nivel representa la comuna (o entidad que toma decisiónes sobre las ciclovías) y maximiza la cantidad de usuarios que utilizan la bicicleta por medio de la decisión de la ubicación y tipo de ciclovías para cada arco de la red. El segundo nivel representa a los usuarios y resuelve el problema del costo del camino más corto para cada par de nodos origen-destino. En principio los objetivos de los subproblemas no son directamente agregables ni es posible solucionar este problema tal cual está con un solver por lo que se estudiarán maneras de reformularlo y formulaciones alternativas.

  Las hipótesis que se asumieron para este trabajo son:

  \begin{itemize}
    \item{El tiempo de viaje en todo arco de la red es independiente del flujo sobre el mismo.}
    \item{Existen diferentes pares origen-destino de demanda en la red y para cada par origen-destino existen diferentes caminos que unen el origen y el destino.}
    \item{Los usuarios siempre buscan minimizar el costo de su viaje (todos son
    perfectos optimizadores y se comportan igual)}
  \end{itemize}

  \subsection*{Formulación matemática}

  Sean los siguientes conjuntos y parámetros:

  \begin{description}
    \item[$OD$]: Conjunto de pares origen destino con elemento genérico $k$.
    \item[$G=(N,A)$]: Grafo dirigido que modela la red, con su conjunto de nodos $N$ y de arcos $A$. $A_n^+$ y $A_n^-$ son los conjuntos de arcos que salen y entran respectivamente desde el nodo $n$.
    \item[$I$]: Conjunto de infraestructuras de ciclovías.
    \item[$C_{ai}$]: Costo de usuario de atravesar el arco $a \in A$ utilizando la infraestructura $i \in I$. $C_{ai} > 0$.
    \item[$H_{ai}$]: Costo de construcción de la infraestructura $i \in I$ en el arco $a \in A$. $H_{ai} \geq 0$ .
    \item[$B$]: Presupuesto para la construcción de infraestructuras.
    \item[$\theta_{nk}$]: Parámetro que vale 1 si n es el origen del par origen-destino $k \in OD$, -1 si es el destino y 0 en otro caso.
    \item[$w_k$]: Variable de primer nivel que contiene el valor del camino más corto para el par origen destino $k \in OD$.
    \item[$y_{ai}$]: Variable binaria de primer nivel que determina si la infraestructura $i \in I$ está activa en el arco $a \in A$.
    \item[$x_{ak}$]: Variable de segundo nivel que determina si el arco $a \in A$ es parte del camino más corto para el par origen-destino $k \in OD$.
    \item[$f_k$]: Función que determina la demanda que utiliza la bicicleta como modo de transporte en función del costo del camino mas corto.
  \end{description}

  \begin{align}
    \text{max}    & \sum_{k \in OD} f_k(w_k)                                                         & \label{eq:objective1lvl} \\
    \text{s.t.}\; & w_k = \sum_{a \in A} \sum_{k \in OD} \sum_{i \in I} C_{ai}y_{ai}x_{ak}           & \forall k \in OD \label{eq:shortestpath} \\
                  & B \geq \sum_{a \in A} \sum_{i \in I} H_{ai}y_{ai}                                & \label{eq:respectbudget} \\
                  & 1 = \sum_{i \in I} y_{ai}                                                        & \forall a \in A \label{eq:alwaysoney} \\
                  & w_k \geq 0, y_{ai} \in \{0,1\}                                                   & \nonumber \\
                  & \text{min} \sum_{a \in A} \sum_{k \in OD} \sum_{i \in I} C_{ai}y_{ai}x_{ak}      & \label{eq:subproblem} \\
                  & \text{s.t.} \sum_{a \in A_n^+} x_{ak} - \sum_{a \in A_n^-} x_{ak} = \theta_{nk}  & \forall n \in N, k \in OD \label {eq:flowbalance} \\
                  & \modelspace x_{ak} \geq 0                                                        & \nonumber
  \end{align}

  Donde:

  \begin{description}
    \item[\ref{eq:objective1lvl}]: La función objetivo es la suma de los valores de demandas que cambiaron de modo de transporte.
    \item[\ref{eq:shortestpath}]: Restricción que determina el costo del camino más corto dado en el primer nivel.
    \item[\ref{eq:respectbudget}]: Restricción de presupuesto sobre lo que se puede construir.
    \item[\ref{eq:alwaysoney}]: Restricción que requiere que una infraestructura estar activa en cada arco. Se discutirá más sobre esto en la siguiente sección.
    \item[\ref{eq:subproblem}]: Función objetivo del segundo nivel.
    \item[\ref{eq:flowbalance}]: Restricción de balance de flujo.
  \end{description}

  \subsubsection*{Observaciones}

  La linealidad de la formulación propuesta es determinada por las funciónes $f_k$. En las ecuaciones (\ref{eq:shortestpath}) y (\ref{eq:subproblem}) las variables de primer y segundo nivel se multiplican pero dado que las variables de un nivel son parámetro en el otro nivel no se pierde la linealidad de dichas ecuaciónes.

  En la formulación del problema de primer nivel, la restricción (\ref{eq:alwaysoney}) pudo haber sido escrita de manera que al menos una infraestructura esté activa. Esto se puede ver de diferentes maneras, pensando en la realidad modelada, un ciclista podría pasar prácticamente por cualquier calle sin problemas, entonces para que las instancias del modelo sean semánticamente correctas debería existir una infraestructura $i_0 \in I$ cuyo costo de construcción $H_{ai_0}$ sea 0 en todos de los arcos $a \in A$, mas allá de que el costo de usuario pueda ser altísimo. Desde un punto de vista formal, como se verá mas adelante, si no se requiere que en cada arco haya una infraestructura activa, entonces hay que agregar al problema de segundo nivel una restricción que evite que se activen flujos en arcos donde no hay infraestructura activa, es decir: $x_{ak} \leq \sum_{i \in I} y_{ai}, \forall a \in A, k \in OD$. Con esta restricción, el problema de segundo nivel puede no tener solución factible cuando las infraestructuras seleccionadas por el primer nivel no inducen un subgrafo que conectan todos los pares origen-destino.

  Se asumirá de aquí en adelante que las instancias del problema están bien definidas, esto significa que:

  \begin{enumerate}
    \item {$G$ es conexo}
    \item {$\forall k \in OD$ existe un camino $S_k \in G$ con costo de construcción cero, es decir $\sum_{a \in S_k} H_{ai_0} = 0$}
  \end{enumerate}

  \subsection*{Validación de la formulación}

  Para discutir la validez del modelado se utilizará una formulación estándar de problema binivel o BLPP (Bi Level Programming Problem), como se puede encontrara en Bard (\ref{bardbook}).
  Sea la siguiente formulación simplificada del problema:

  \begin{align}
    \text{max}_{y \in Y}    & \; F(x, y) \\
    \text{s.t.} \modelspace & A_1 x + B_1 y \leq b_1 \\
                            & \text{min}_{x \in X} f(x, y) \\
                            & \modelspace A_2 x + B_2 y \leq b_2
  \end{align}

  Y las siguientes definiciones:

  \begin{definition}
    Conjunto factible
    \begin{align}
      S = \{(x, y) \setminus x \in X, y \in Y, A_1 x + B_1 y \leq b_1, A_2 x + B_2 y \leq b_2 \}
    \end{align}
  \end{definition}

  \begin{definition}
    Conjunto de reacción del segundo nivel:
    \begin{align}
      P(y) = \{ x \in \text{argmin}_{\hat{x} \in X} f(\hat{x}, y) : A_2 \hat{x} + B_2 y \leq b_2 \}
    \end{align}
  \end{definition}

  Diremos que el problema está bien formulado si el conjunto $S$ es no vacío, es decir, si existen soluciones factibles y si para toda $y$ el conjunto $P(y)$ es no vacío, es decir, si para todo movimiento del problema de primer nivel, hay margen de decisión en el segundo nivel.

  \begin{lemma}$S$ es no vacío
  \end{lemma}

  \begin{proof}
    $S \neq \emptyset$ ya que $\exists (x_0, y_0) \in X \times Y$ donde $y_0$ es el vector $y_{ai_0} = 1 \forall a \in A$, $i_0$ es la infraestructura cuyo costo $H_{ai_0} = 0$, lo que deja al resto de las entradas del vector $y_0$ en $0$. Por lo tanto se cumple las restricción (\ref{eq:alwaysoney}) dado que todos los arcos tienen una infraestructura activa, y la restricción (\ref{eq:respectbudget}) dado que el costo total de construcción de $y_0$ es $0$.

    Luego, dado que las infraestructuras activas logran la conectividad de los pares origen-destino y el hecho de que el el costo de los arcos $C_ai$ sea no negativo permite asegurar que el problema de segundo nivel tiene al menos una solución factible $x_0$.
  \end{proof}

  \begin{lemma}$\forall y \in Y,\; P(y) \neq \emptyset$
  \end{lemma}

  \begin{proof}
    Para cualquier asignación $y = \hat{y} \in Y$, se cumple que $P(\hat{y})$ es no vacío ya que todos los arcos tienen una infraestructura activa, por lo tanto el grafo donde los flujos del problema de segundo nivel pueden pasar es conexo y llega necesariamente a todos los nodos, incluyendo los pares origen-destino. Por lo tanto el espacio de soluciones factibles del subproblema es no vacío. 
  \end{proof}

  Aún cuando estas dos definiciones se cumplen, puede haber dificultades al encontrar la solución óptima cuando $P(y)$ contiene más de un elemento, caso en el cual el modelo puede no llegar a la solución óptima, dependiendo del valor $x \in P(y)$ que seleccione el problema de segundo nivel. En este caso de estudio esto es muy probable que suceda, puesto que pueden haber varios caminos más cortos entre dos puntos.

  \subsection*{Formulación alternativa de un nivel}

  Si bien la formulación binivel expresa de buena manera lo que se quiere resolver, en la práctica los BLPP son en general de dificil resolución. Por eso, antes de continuar con su el estudio, se analizará una formulación alternativa como LP de un nivel.

  Esta formulación nace de la formulación binivel quitando el subproblema y agregando sus restricciones al problema de primer nivel. El argumento para hacer esto es que las funciones $f_k$ deben ser decrecientes, entonces para maximizar $\sum_{j \in OD}f_k(w_k)$ lo mejor es los $w_k$ sean lo más chico posible.

  La formulación es la siguiente:

  \begin{align}
    \text{max}    & \sum_{k \in OD} f_k(w_k)                                                         & \label{eq:objectivealt} \\
    \text{s.t.}\; & w_k = \sum_{a \in A} \sum_{k \in OD} \sum_{i \in I} C_{ai}y_{ai}x_{ak}           & \forall k \in OD \label{eq:shortestpathalt} \\
                  & B \geq \sum_{a \in A} \sum_{i \in I} H_{ai}y_{ai}                                & \label{eq:respectbudgetalt} \\
                  & 1 = \sum_{i \in I} y_{ai}                                                        & \forall a \in A \label{eq:alwaysoneyalt} \\
                  & \sum_{a \in A_n^+} x_{ak} - \sum_{a \in A_n^-} x_{ak} = \theta_{nk}              & \forall n \in N, k \in OD \label{eq:flowbalancealt} \\
                  & w_k \geq 0, x_{ak} \geq 0, y_{ai} \in \{0,1\}                                    & \nonumber
  \end{align}

  \subsubsection*{Observaciones}

  Inicialmente se había planteado este mismo modelo pero con una función objetivo distinta. La idea era llevar la función objetivo de segundo nivel de la formulación inicial al primer nivel logrando una función multiobjetivo. Luego, como la unidad del primer nivel es unidades demanda y la del segundo nivel costo de usuario, se intentan unificar ambas multiplicando el primer nivel por el costo de usuario y el segundo por demanda quedando el siguiente objetivo: $min\;\sum_{k \in K} D_kw_k - f_k(w_k)w_k$. La idea es minimizar el costo del camino más corto $w_k$ y maximizar $f_k$ (minimizar el opuesto). Este planteo tiene la desventaja de tener la multiplicación de variables que seguramente sea posible circunvalar a razón de mayor complejidad de la formulación.

  \section*{Resolución del problema BLPP}

  Al igual que en su versión de un nivel, la mayoría de los avances en algorítmos de resolución se ha centrado en la versión lineal. Ya se ha demostrado en Bard (\ref{bardbook}) que el BLPP lineal es NP-Hard.

  \subsection*{Transformación de binivel a un nivel}

  Este problema se puede transformar a un problema de un nivel sustituyendo el problema de segundo nivel por sus restricciónes de optimalidad de KKT en el primer nivel. De obtenerse una representación lineal se podría buscar alguna metodología exacta para resolverlo. La formulación binivel estudiada presenta el problema, a los efectos de la transformación, de contener ecuaciones con variables de diferentes niveles multiplicándose, dichas ecuaciones deben ser reformuladas para que no devengan en restricciones no lineales.

  \subsubsection*{Quitando multiplicación de variables}

  El objetivo es quitar la multiplicación entre $x_{ak}$ e $y_{ai}$. Se propone la formulación siguiente del problema de segundo nivel.

  \begin{align}
    \text{min}  & \sum_{k \in K} \sum_{a \in A} \sum_{i \in I} C_{ai} h_{aki}         & \label{eq:subproblemrefeq1} \\
    \text{s.t.} & \sum_{a \in A_n^+} x_{ak} - \sum_{a \in A_n^-} x_{ak} = \theta_{nk} & \forall n \in N, k \in OD \\
                & 0 \leq h_{aki} \leq y_{ai}                                          & \forall a \in A, k \in K, i \in I \\
                & x_{ak} = \sum_{i \in I} h_{aki}                                     & \forall a \in A, k \in K
  \end{align}

  Si utilizamos la ecuación (\ref{eq:subproblemrefeq1}) en la restricción de igualdad de $w_k$ (ecuación \ref{eq:shortestpath}) el problema es equivalente con la ventaja que no existen variables de diferentes niveles multiplicándose. La equivalencia se da porque esta formulación desagrega los flujos para cada infraestructura ademas de par origen-destino y arco en la variable $h_{aki}$, que es lo que se modela con con $y_{ai} x_{ak}$.

  \subsection*{Definición de las $f_k$s}

  Estas funciones fueron dejadas de lado en la formulación inicial con el objetivo de analizar las restricciones principales primero. Se analizan en esta sección diferentes alternativas de cómo implementarlas como un conjunto de ecuaciones lineales que se acoplaran a las formulación antedicha.

  Podemos encontrar en la literatura soluciones a problemas similares. En Laporte 2007 (\ref{laporte2007}) se considera un parámetro $c^{PRIV}_k$ que modela el costo del transporte privado para un par origen-destino $k$, si la red de transporte público logra un costo menor entonces se considera que toda la demanda se transfiere a dicho modo.

  \subsubsection*{Definición propuesta}

  Se considera que una función $f_k$ arbitraria debe poder modelar una transición paulatina de la demanda entre modos de transporte. Paulatina de manera de expresar una transición que pueda parecerse a algo lineal, exponencial o lo que fuere. Asumiendo que las $f_k$ son decrecientes, la mejor forma que ocurriese es representarla como una sucesión decreciente de puntos de quiebre, tal que para cada uno se exista un valor asociado de cantidad de demanda transferida. Los puntos de quiebre son comparados contra el valor del camino más corto $w_k$. Entonces sea $w_k$ fijo, la demanda transferida es:

  \begin{equation}
    \label{eq:deffks}
    f_k(w_k) = P_{j^*},\; j^* = argmin_{j \in J} \{Q_j \geq w_k\}
  \end{equation}
  
  Donde $Q_j$ son los puntos de quiebre en la unidad del costo del camino más corto, $J$ es un conjunto índice y $P_{j^*}$ es la cantidad de demanda que se transfiere. Una ventaja de esta formulación es que puede ser integrada a la función objetivo de la formulación inicial de manera que la minimización queda en manos del objetivo del mismo problema.

  Como alternativa, se puede podría pensar una definición de $f_k$ como función no lineal, esto puede llevar a soluciones soluciones analíticamente más precisas pero es sabido que aumenta considerablemente la dificultad de resolución práctica.

  Para modelar la función $argmin$ en una formulación como la que se viene trabajando es necesario utilizar variables de activación que solo permita que una entrada del índice esté activo. Por ejemplo si se usa $z_j,\; j \in J$, lo antedicho equivale a decir que $z_j \in \{0,1\}$ y $\sum_{j \in J} z_j = 1$. Como estas restricciones se encuentran dentro de una maximización y suponiendo que el conjunto ordenado $Q$ es estrictamente decrecientes, podemos relajar la restricción de integralidad y sustituirla por $0 \leq z_j \leq 1$. Es esperable pensar que no pueden existir varios $z_j$ activos para la definición (\ref{eq:deffks}), porque si los hubiesen, al ser $Q$ y $P$ conjuntos ordenados estrictamente decreciente entonces no se llegaría al máximo.

  Nótese que esta definición es una extensión de la propuesta por Laporte 2007 (\ref{laporte2007}). Para modelarla basta considerar $J = \{0, 1\}$, $P_0 = 0, P_1 = D_k$ y $Q_0 = M, Q_1 = c^{PRIV}_k$, donde $M$ es un numero arbitrario mayor a $c^{PRIV}_k$ y $D_k$ es la demanda para el par origen-destino $k$.

  \subsubsection*{Formulación de $f_k$ como LP}

  En esta sección se discuten diferentes formulaciones de $f_k$ como LP con el objetivo de analizar cómo se integrará a la formulación del problema que se quiere resolver. La forma más simple de modelar $f_k$ (ó simplemente $f$) es la (\ref{eq:fkv1eq1})-(\ref{eq:fkv1eq4}).

  \begin{align}
    f(W) =\; & max \sum_{j \in J} P_j y_j    & \label{eq:fkv1eq1}\\
             & s.t. \sum_{j \in J} y_j = 1   & \label{eq:fkv1eq2} \\
             & \;\;\; Q_j \geq W y_j         & \label{eq:fkv1eq3} \forall j \in J \\
             & \;\;\; y_j \in \{0,1\}        & \label{eq:fkv1eq4}
  \end{align}

  Si se intenta integrar esta formulación a (\ref{eq:objective1lvlfinal})-(\ref{eq:respectinfra}) el resultado seria un problema quadrático ya que $W$ sería sustituido por $w_k$. Por este motivo se estudiaron dos formulaciones alternativas que logran solucionar esta situación. La idea en ambas es desagregar $W$ como la suma de nuevas variables y utilizar estas en la restricción (\ref{eq:fkv1eq3}) en lugar de $W y_j$.

  \paragraph*{$f_k$ como LP version 1}

  \begin{align}
    f(W) =\; & max \sum_{j \in J} P_j y_j             & \label{eq:fkv3eq1}\\
             & s.t. \sum_{j \in J} y_j = 1            & \label{eq:fkv3eq2}\\
             & \;\;\; Q_j \geq w^{aux}_j              & \forall j \in J \label{eq:fkv3eq3} \\
             & \;\;\; w^{aux}_j \leq M y_j            & \forall j \in J \label{eq:fkv3eq4} \\
             & \;\;\; w^{sink}_j \leq M (1 - y_j)     & \forall j \in J \label{eq:fkv3eq5} \\
             & \;\;\; w^{sink}_j + w^{aux}_j = W      & \label{eq:fkv3eq6} \\
             & \;\;\; y_j \in \{0,1\}                 & \label{eq:fkv3domainy} \\
             & \;\;\; w^{aux}_j, w^{sink}_j \geq 0    & \label{eq:fkv3eq7}
  \end{align}

  Se desagrega $W$, para cada $j$ en $w^{sink}_j + w^{aux}_j$ de manera que uno de los dos tenga el valor de $W$, esto se logra con las restricciones (\ref{eq:fkv3eq4}) y (\ref{eq:fkv3eq5}). Entonces se cumple que solo el $w^{aux}_j$ cuyo $y_j$ esté activo tendrá el valor de $W$. Las mencionadas restricciones sirven para activar $y_i$ utilizado un parámetro $M \geq W$ arbitrario. Finalmente las restricciónes (\ref{eq:fkv3eq2}) y (\ref{eq:fkv3eq6}) logran que solo una de las variables $w^{aux}_j$ o $w^{sink}_j$ tome el valor de $W$, para cada $j$.

  \paragraph*{$f_k$ como LP version 2}

  \begin{align}
    f(W) =\; & max \sum_{j \in J} P_j y_j             & \label{eq:fkv2eq1}\\
             & s.t. \sum_{j \in J} y_j = 1            & \label{eq:fkv2eq2}\\
             & \;\;\; Q_j \geq w^{aux}_j              & \forall j \in J \label{eq:implfkoriginalineq} \\
             & \;\;\; w^{aux}_j \leq M y_j            & \forall j \in J \label{eq:yactivation1} \\
             & \;\;\; \sum_{j \in J} w^{aux}_j = W    & \label{eq:activatewaux} \\
             & \;\;\; y_j \in \{0,1\}                 & \label{eq:fkv2domainy} \\
             & \;\;\; w^{aux}_j \geq 0                & \label{eq:fkv2eq6}
  \end{align}

  En este caso se desagrega $W$ como suma de $|J|$ variables $w^{aux}_j$ de manera que solo una de ellas esté activa y tenga el valor de $W$. Para activar $y_i$ se agrega la restricción (\ref{eq:yactivation1}) utilizado, igual que en la formulación anterior, un parámetro $M \geq W$ y finalmente las restricciónes (\ref{eq:activatewaux}) y (\ref{eq:fkv2eq2}) logran que solo uno de los $w^{aux}_j$ tome el valor de $W$ y el resto sean 0.

  \subsubsection*{Validación de la formulación}

  Según definimos $f_k(W)$ en (\ref{eq:deffks}), sea de $p = f_k(W)$, entonces se cumple que:

  \begin{enumerate}
    \item {\label{deffpt1} $p$ es exactamente uno de los valores del conjunto de valores $P_j$, es decir: $p \in \{P_j\}_{j \in J}$}
    \item {\label{deffpt2} Sea $\hat{j} = j \;|\; p = P_j$. El valor del punto de quiebre $Q$ asociado a $\hat{j}$ no es menor que el costo del camino más corto $W$, es decir: $Q_{\hat{j}} \geq W$.}
    \item {\label{deffpt3} Dado el $\hat{j}$ anterior, entonces no existe un punto de quiebre $Q_j \forall j \in J$ que sea no mayor al punto de quiebre $Q_{\hat{j}}$ y mayor o igual al costo del camino más corto, es decir: $\not{\exists}\; Q_j\; \forall j \in J \;|\; Q_j \in  [W, Q_{\hat{j}}]$}.
  \end{enumerate}

  Es de interés analizar la correctitud de las formulaciones respecto a los puntos mencionados. En particular, la versión 2 (\ref{eq:fkv2eq1}) -(\ref{eq:fkv2eq6}) que es la que se utilizará finalmente. Se tiene que el punto \ref{deffpt1} se cumple, en la formulación, en la función objetivo (\ref{eq:fkv2eq1}) donde el valor objetivo es la suma de un subconjunto de $\{P_j\}_{j \in J}$, la restricción (\ref{eq:fkv2eq2}) que dice que solo uno de esos valores puede estar activo en la suma del objetivo y la restricción (\ref{eq:fkv2domainy}) que determina el carácter de la variable de decisión. El punto \ref{deffpt2} se cumple en la restricción (\ref{eq:implfkoriginalineq}), dado que $w^{aux}_j$ es igual a $W$ o cero, como se detalla en las restricciones (\ref{eq:yactivation1}), (\ref{eq:activatewaux}) y el carácter binario de $y_j$. Si $w^{aux}_j = W$, entonces el $j$ activo es el valor funcional de $f(W)$ y se cumple la definición del punto \ref{deffpt2}. Por otro lado, si es cero entonces el $j$ activo no es $\hat{j}$ y por lo tanto también se cumple el punto en cuestión. Finalmente, el punto \ref{deffpt3} se cumple por el hecho de que la función objetivo es una maximización y que $f(W)$ es una función estrictamente decreciente, entonces si existiera un $Q_a \in \{Q_j\}_{j \in J} \;|\; Q_a < Q_{\hat{j}}, a \in J$ entonces por se $f$ decreciente existe un valor $P_a > P_{\hat{j}}$ lo que contradice el hecho mismo de que el objetivo sea una maximización.

  \subsubsection*{Valores posibles de $P_j$ y $Q_j$}

  Para un $k \in OD$, sea $\overline{W}$ el costo del camino más corto sobre el grafo base (sin infraestructuras) y $\underline{W}$ el mejor costo del camino más corto suponiendo que las mejores infraestructuras se pueden construir, entonces $Q_j \in [\underline{W}, \overline{W}],\; \forall j \in J$\footnote{En realidad, $\overline{W}$ es el mínimo valor de la para el que $f_k$ es mínimo, podría ser cualquier numero arbitrariamente grande.}. Los valores de $Pj$ son la cantidad de demanda que se transfiere para cada $Q_j$ y dependen del par origen-destino $k$.

  Una forma posible de establecer los valores de $J$, $P_j$ y $Q_J$ es, dada una aproximación continua de $f_k$, tomando $N$ valores equidistantes entre $[\underline{W}, \overline{W}]$ obtendremos $Q_j$, y tomando el valor funcional de esos puntos obtendremos $P_j$, luego $J=\{1,..,N\}$. Es deseable también que el mínimo de $f_k$ sea 0, es decir, que si ninguna mejora es impuesta al camino más corto no haya transferencia de demanda.

  \section*{Poniendo todo junto}

  Se escribe aquí las formulaciónes completas del problema en su forma binivel y su alternativa de un nivel, agregando las definiciones de $f_k$ y quitando la multiplicación de variables.

  \subsection*{Formulación binivel}

  Además de las definiciones en la formulación inicial (\ref{eq:objective1lvl})-(\ref{eq:flowbalance}):

  \begin{description}
    \item[$J$]: Es un conjunto índice utilizado en los conjuntos $P$ y $Q$.
    \item[$P_{kj}$]: Parámetro que determina la cantidad de demanda transferida para el par origen-destino $k \in OD$ y el índice $j \in J$.
    \item[$Q_{kj}$]: Parámetro que contiene el punto de quiebre para determinar la demanda transferida para el par origen-destino $k \in OD$ e índice $j \in J$.
    \item[$M$]: Número positivo muy grande. 
    \item[$z_{kj}$]: Variable binaria que determina si demanda transferida para el par origen-destino $k \in OD$ es la de índice $j \in J$.
    \item[$h_{aki}$]: Variable no negativa que determina el flujo que pasa por el arco $a \in A$, para el par origen-destino $k \in OD$ utilizando la infraestructura $i \in I$.
    \item[$w^{aux}_{kj}$]: Variable no negativa que contiene el valor de $w_{k}$ si $z_{kj}$ esta activa y cero sino. 
  \end{description}

  \begin{align}
    \text{max}    & \sum_{k \in OD} \sum_{j \in J} P_{kj} z_{kj}                                     & \label{eq:objective1lvlfinal} \\
    \text{s.t.}\; & w_k = \sum_{a \in A} \sum_{i \in I} C_{ai}h_{aki}                                & \forall k \in OD \label{eq:shortestpathfinal} \\
                  & Q_{kj} \geq w^{aux}_{kj}                                                         & \forall j \in J, k \in OD \label{eq:breakpoints} \\
                  & w^{aux}_{kj} \leq M z_{kj}                                                       & \forall j \in J, k \in OD \\
                  & \sum_{j \in J} w^{aux}_{kj} = w_k                                                & \forall k \in OD \\
                  & \sum_{j \in J} z_{kj} = 1                                                        & \forall k \in OD \label{eq:singularbreakpoint} \\
                  & B \geq \sum_{a \in A} \sum_{i \in I} H_{ai}y_{ai}                                & \label{eq:respectbudgetfinal} \\
                  & 1 = \sum_{i \in I} y_{ai}                                                        & \forall a \in A \label{eq:alwaysoneyfinal} \\
                  & w_k \geq 0, y_{ai} \in \{0,1\}, z_{kj} \in \{0,1\}                               & \nonumber \\
                  & \text{min} \sum_{k \in K} \sum_{a \in A} \sum_{i \in I} C_{ai} h_{aki}           & \label{eq:subproblemfinal} \\
                  & \text{s.t.} \sum_{a \in A_n^+} x_{ak} - \sum_{a \in A_n^-} x_{ak} = \theta_{nk}  & \forall n \in N, k \in OD \label{eq:flowbalancefinal} \\
                  & \modelspace x_{ak} = \sum_{i \in I} h_{aki}                                      & \forall a \in A, k \in K \label{eq:flowactivation} \\
                  & \modelspace 0 \leq h_{aki} \leq y_{ai}                                           & \forall a \in A, k \in K, i \in I \label{eq:respectinfra} \\
                  & \modelspace x_{ak} \geq 0, h_{aki} \geq 0                                        & \forall a \in A, k \in K \nonumber
  \end{align}

  Donde las nuevas ecuaciones son:

  \begin{description}
    \item[\ref{eq:objective1lvlfinal}]: Función que suma los valores de demanda transferida $P_{kj}$ activos.
    \item[\ref{eq:breakpoints}]: Restricción que determina que los puntos de quiebre activos son aquellos cuyo costo es menor o igual al del camino más corto.
    \item[\ref{eq:singularbreakpoint}]: Restricción que permite solo un punto de quiebre activo para cada par origen-destino $k \in OD$.
    \item[\ref{eq:flowactivation}]: El flujo total para el arco $a \in A$ y el par origen-destino $k \in OD$ es la suma de los flujos de todos las infraestructuras.
    \item[\ref{eq:respectinfra}]: El flujo por el arco $a \in A$, para el par origen-destino $k \in OD$ y la infraestructura $i \in I$ puede estar activo si la infraestructura $i$ esta activa.  
  \end{description}

  \subsection*{Formulación de un nivel}

  \begin{align}
    \text{max}    & \sum_{k \in OD} \sum_{j \in J} P_{kj} z_{kj}                          & \label{eq:objectivealtfinal} \\
    \text{s.t.}\; & w_k = \sum_{a \in A} \sum_{i \in I} C_{ai}h_{aki}                     & \forall k \in OD \label{eq:shortestpathaltfinal} \\
                  & Q_{kj} \geq w^{aux}_{kj}                                              & \forall j \in J, k \in OD \label{eq:breakpointsalt} \\
                  & w^{aux}_{kj} \leq M z_{kj}                                            & \forall j \in J, k \in OD \\
                  & \sum_{j \in J} w^{aux}_{kj} = w_k                                     & \forall k \in OD \\
                  & \sum_{j \in J} z_{kj} = 1                                             & \forall k \in OD \label{eq:singularbreakpointalt} \\
                  & B \geq \sum_{a \in A} \sum_{i \in I} H_{ai}y_{ai}                     & \label{eq:respectbudgetaltfinal} \\
                  & 1 = \sum_{i \in I} y_{ai}                                             & \forall a \in A \label{eq:alwaysoneyaltfinal} \\
                  & \sum_{a \in A_n^+} x_{ak} - \sum_{a \in A_n^-} x_{ak} = \theta_{nk}   & \forall n \in N, k \in OD \label{eq:flowbalancealtfinal} \\
                  & x_{ak} = \sum_{i \in I} h_{aki}                                       & \forall a \in A, k \in K \label{eq:flowactivationalt} \\
                  & 0 \leq h_{aki} \leq y_{ai}                                            & \forall a \in A, k \in K, i \in I \label{eq:respectinfraalt} \\
                  & z_{kj} \in \{0,1\}                                                    & \nonumber \\
                  & w_k \geq 0, y_{ai} \in \{0,1\}, x_{ak} \geq 0, h_{aki} \geq 0         & \nonumber
  \end{align}

  Donde las definiciones de conjuntos, parámetros y variables son equivalentes a la formulación (\ref{eq:objective1lvlfinal})-(\ref{eq:respectinfra}).

  \section*{Características de una solución}

  Como manera de validar de alguna manera las formulaciones de la sección anterior se detallaran las características deseables de una solución y en qué ecuaciones de las formulaciones son tenidas en cuenta.

  Una solución al problema:

  \begin{enumerate}
    \item{El costo de los caminos entre pares origen-destino sobre la red resultante es menor o igual al costo sobre la red sin infraestructuras.}
    \item{\label{budgetexcess} El presupuesto excedente no es suficiente para agregar una infraestructura que mejore el costo de alguno de los caminos.}
    \item{El camino más corto sobre la red resultante para un par origen-destino no puede inducir un valor de demanda transferida distinto al de la solución.}
  \end{enumerate}

  \section*{Notas de implementación}

  \subsection*{Versión de un nivel}

  Esta formulación funcionó bien a los efectos de calcular la demanda transferida y ciertos flujos óptimos, pero sucedió en algunos casos que el modelo no calculaba el flujo óptimo para cierto par origen destino. Esto se detecto al observar que los valores de algunos flujos no eran unitarios y en que el costo del camino mas corto era igual al de algún punto de quiebre $Q_{kj}$. Para solucionar esto, se agregó otra variable de holgura $r_{kj} \geq 0$ que modela la diferencia entre el punto de quiebre $Q_{kj}$ activo y el costo del camino mas corto $w_k$. Luego se agregó dicha variable a la función objetivo de manera que se incentivar el crecimiento de esta variable. La ecuación (\ref{eq:implfkoriginalineq}), que en el modelo final agrega los indices por par origen-destino $k$, quedó entonces de esta manera:

  \begin{equation}
    Q_{kj} z_{kj} - r_{kj} = w^{aux}_{kj}
  \end{equation}

  Luego, la función objetivo:

  \begin{equation}
    \sum_{k \in OD} \sum_{j \in J} P_{kj}z_{kj} + r_{kj}
  \end{equation}

  Resta por determinar los efectos de esta variable en la función objetivo. Otro enfoque es dejar el calculo del camino mas corto para cada par origen-destino para una etapa de postprocesamiento de la solución.

  La relajación de la integralidad de las variables $z_{kj}$, como se menciona en la sección de definición de $f_k$ debe seguir siendo estudiada ya que dependiendo del solver, la instancia y la forma en que se modele $f_k$ los resultados pueden variar.

  \section*{Especificación de Datos}

  Para facilitar la ejecución de instancias, se desarrollo una pequeña biblioteca que permite la especificación de una red y sus datos asociados de manera prográmatica para luego exportarla al tipo de datos que el modelo soportado por el solver.

  Las redes analizadas deben tener una serie de datos asociados de manera que sea posible solucionarlas con el modelo planteado. Para los nodos es necesario (no estrictamente) tener su par de coordenadas, para los arcos, se necesitan los siguientes atributos:

  \begin{description}
    \item[distance]: Distancia o longitud del arco.
    \item[user\_cost]: Costo de usuario de atravesar el arco sobre el grafo base (sin infraestructura o con la infraestructura base $i_0$).
    \item[construction\_cost]: Costo de construcción de la infraestructura base.
  \end{description}

  Luego, si se tienen $N$ infraestructura, para la infraestructura $n$ se calcula el costo de usuario de atravesarla como $user\_cost (-3 (n + 1) + 28) / 25$. El valor del costo de construcción será $2 n construction\_cost$. De ser necesario, pueden especificarse valores particulares de costos de dicha infraestructura para un arco mediante la utilización de los atributos $user\_cost\_n$ y $construction\_cost\_n$.

  \section*{Validación de implementación}

  Para validar y testear la implementación se implementaron, valga la redundancia, chequeos que verifican las ya mencionadas características de una solución. Se probaron diferentes alternativas de la formulación, las variantes probadas fueron: dos implementaciónes de las $f_k$ y presencia o no de las variables de holgura $r_{kj}$.

  Se utilizó la ciudad de Sioux-Falls como grafo base, dada su fama en la academia. Los datos fueron obtenidos del repositorio de instancias de redes de transporte (\ref{transportationnetworkrepo}), de ahi se tomaron los datos de posición de nodos y largo de los arcos. El costo de usuario se tomo igual al largo del arco y así como el costo de construcción de la primer infraestructura no base.

  Se generaron pares origen-destino y valores de $P$ y $Q$ aleatorios. De esta manera se generaron y probaron 1000 instancias para cada uno de las cuatro variantes de las formulaciones. Ademas de los chequeos sobre las soluciones, es de interés comparar los valores de demand transferida total y tiempo de ejecución.

  De las ejecuciones, cuyo resumen se encuentra en el cuadro (\ref{table:resumenejecuciones}) se pudo comprobar que el modelo con la primera formulación de $f_k$ tuvieron tiempo de ejecución mucho mayor que aquellos con la segunda version.

  \begin{table}[h!]
    \centering
    \caption*{{\bf Resumen de ejecuciones}}
    \begin{tabular}{ccccc}
      \toprule
      Version & Version de $f_k$ & Variables $r$ & Cant. Errores & Tiempo promedio (s) \\
      v1 & v2 & Si & 1  & 5.06   \\
      v2 & v2 & No & 73 & 8.28   \\
      v3 & v1 & Si & 1  & 157.63 \\
      v4 & v1 & No & 48 & 170.63 \\
      \midrule
      \bottomrule
    \end{tabular}
    \caption{Comparativa agregada de ejecuciones sobre instancias aleatorias utilizando el la red Sioux-Falls. Los errores encontrados para los modelos v1 y v3 se determinaron para una misma instancia y la razón fue un problema numérico que también afectó a las otras versiones de las formulaciones sobre la misma instancia. Por otro lado, el grueso de los errores en las versiones v2 y v4 se deben al incumplimiento del punto (\ref{budgetexcess}) de las características de una solución deseable.}\label{table:resumenejecuciones}
  \end{table}

  % TODO: datos, gráficas, cosos sobre las ejecuciones.

  En las pruebas, se detecto que la ausencia de variables de holgura $r$ no afecta el valor de la demanda transferida total en la mayoría de los casos. Se encontró que en las versiones que utilizan variables $r$ y se cumple que el valor de demanda transferida en para un par origen-destino es comparable a la magnitud de su $r$ correspondiente, entonces si se puede afectar la solución final. Por ejemplo, si para un par origen destino se puede optimizar el camino más corto de manera de ganar una unidad de demanda transferida pero esto afecta las variables $r_{kj}$ de todos los pares origen destino de manera negativa, entonces el modelo puede elegir no realizar dicha optimización.
  
  Una observación sobre las versiones sin variables $r$ es que se ve afectada en algunos casos la validez del item (\ref{budgetexcess}), dado que, si la construcción de una infraestructura en un arco puede mejorar el costo del camino más corto pero no lo suficiente para afectar el objetivo entonces puede quedar presupuesto excedente. Esto implica también que las infraestructuras activas sin utilizar variables $r$ es menos óptima. Esto se puede explicar por ejemplo, para un par origen-destino y dos puntos de quiebre consecutivos, pueden haber varios conjuntos de infraestructuras que logren costo de camino más corto entre dichos puntos de quiebre, entonces si no se utilizan las variables $r$ no hay incentivos para elegir uno del menor costo.

  Considerando el tiempo de ejecución promedio y el hecho de que no hubo errores en los chequeos se eligió la segunda version de formulación de $f_k$ con variables de holgura, es decir la formulación v1, como formulación final.

  \section*{Resultados experimentales}

  Se probó la formulación final en la instancia de Montevideo de manera de poder analizar su aplicación práctica sobre datos familiares y poder compararlos con trabajos similares enmarcados dentro del proyecto AGESIC "Diseño óptimo de redes de ciclovias".


  % TODO: Descripción de la instancia
  % - arcos, nodos, centroides
  % - matriz demanda: origen datos
  % - datos de demanda transferida: inventados seguramente

  \subsubsection*{Comparación frente a otros problemas}

  TODO: aca comparar los resultados con el otro problema de ciclovias trabajado como parte del proyecto de ciclovias.

  \section*{Referencias}

  \begin{enumerate}
    \item{\label{bardbook} Jonathan F. Bard (1998). Practical Bilevel Optimization, Algorithms and Applications}
    \item{\label{laporte2007} Gilbert Laporte, Ángel Marín, et al. (2007). An Integrated Methodology for the Rapid Transit Network Design Problem}
    \item{\label{transportationnetworkrepo} Transportation Networks Repositort \url{https://github.com/bstabler/TransportationNetworks}}
  \end{enumerate}
\end{document}
