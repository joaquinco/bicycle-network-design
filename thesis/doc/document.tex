\documentclass{article}
  \parindent = 0mm % Sin sangría
  \usepackage[utf8]{inputenc}
  \usepackage[T1]{fontenc}
  \usepackage[spanish]{babel}
  \usepackage{graphicx}
  \usepackage{amstext}
  \usepackage{amsmath}
  \usepackage{booktabs}
  \usepackage{subfigure}
  \usepackage{footnote}
  \usepackage{hyperref}
  \usepackage{algpseudocode,algorithm,algorithmicx}
  \usepackage[font=small,labelfont=bf]{caption}

\begin{document}
  \begin{center}
    {\sc \large Maestría en Investigación de Operaciones}
    
    {\sc \large Tesis - Tentativo}
    \linebreak

    {\rm Joaquín Correa - \today}
  \end{center}

  \section*{Problema}

  \subsection*{Descripción}

  El problema planteado se ubica dentro del contexto de diseño y planificación de ciclovias en una ciudad. La vida es mala.

  Dadas información de demanda en una ciudad, el objetivo es maximizar la cantidad de demanda que utiliza la bicicleta como medio. Para poder lograr esto, se cuenta con infraestructuras y presupuesto que, de ser utilizados, permiten disminuir el costo percibido por el usuario y por lo tanto potencialmente influir la decision de utilizar la bicicleta.

  \subsection*{Formulación}

  
\end{document}
